\documentclass[10pt]{beamer}
\usepackage[utf8]{inputenc}
\usetheme[sectionpage=none,
    subsectionpage=progressbar,
    progressbar=frametitle,
    numbering=fraction,
    block=fill]{metropolis}
\usepackage{xcolor}

\makeatletter
\setlength{\metropolis@titleseparator@linewidth}{2pt}
\setlength{\metropolis@progressonsectionpage@linewidth}{2pt}
\setlength{\metropolis@progressinheadfoot@linewidth}{2pt}
\makeatother

\definecolor{fajny_szary}{HTML}{716b58}
\definecolor{fajny_bar}{HTML}{c2b385}

\setbeamercolor{progress bar}{fg=fajny_bar}
\setbeamercolor{frametitle}{fg=white, bg=fajny_szary}

\usepackage{lipsum}% http://ctan.org/pkg/lipsum
\usepackage{hanging}% http://ctan.org/pkg/hanging
\setbeamertemplate{footnote}{%
  \hangpara{2em}{1}%
  \makebox[2em][l]{\insertfootnotemark}\footnotesize\insertfootnotetext\par%
}

\usepackage{polski}
\usepackage[polish]{babel}
\usepackage{setspace,amsmath,booktabs}

\title[]{Współczesne rozwiązania technologiczne pomagają w rozwoju i edukacji}
\author[A.~Greloch \and W.~Zaremba \and M.~Zwierzyński \and J.~Lipski \and K.~Szturemski]{A.~Greloch \and W.~Zaremba \and M.~Zwierzyński \and J.~Lipski \and K.~Szturemski}
\institute[G1PIA] % (optional)
{
  Klasa 3j\\
  Gimnazjum nr. 1 im. Powstańców Warszawy w Piasecznie
}
\date{Piaseczno, 16 maja 2018}

\begin{document}

\frame{\titlepage}

\section{Problem}
\subsection{Wprowadzenie do problematyki projektu}
\setcounter{subsection}{1}

\renewcommand{\arraystretch}{1.2}
\setstretch{1.1}

\begin{frame}
  \frametitle{Problem}
  \begin{enumerate}
    \item Zbyt duża różnorodność treści w internecie $\rightarrow$ różna jakość i wiarygodność dostępnych informacji
    \item Zbyt duża popularność portali \textbf{typu social-learning}
    \item Za mała popularność rzetelnych internetowych źródeł wiedzy
    \item \textbf{Negatywna opinia o internecie jako medium naukowego}
  \end{enumerate}
  \begin{block}{Platformy social-learning'owe}
    Typ portalu społecznościowego, skupionego na udostępnianiu odpowiedzi do zadań z różnych przedmiotów tj. \emph{zadane.pl}, \emph{brainly.pl}, \emph{sciaga.pl}.
  \end{block}
\end{frame}

\begin{frame}
  \frametitle{Rzetelne internetowe źródła wiedzy}
  \small

  \begin{description}
    \item [\textbf{Platformy e-learning'owe}]\hfill\\ Typ portalu społecznościowego, skupionego na udostępnianiu materiałów edukacyjnych i opracowań z różnych dziedzin. Przykładem takiego portalu jest \emph{e-podreczniki.pl}.
    \item [\textbf{Encyklopedie, e-słowniki, e-biblioteki}]\hfill\\ Typ serwisu internetowego, udostępniającego zinformatyzowaną wersję źródeł naukowych oraz literackich. Przykładami takiego serwisu są \emph{wikipedia.org}, \emph{wolnelektury.net}, \emph{ebuw.uw.edu.pl} \footnote[frame]{e-biblioteka Uniwersytetu Warszawskiego}.
  \end{description}

\end{frame}

\begin{frame}
  \frametitle{Powstawanie negatywnej opinii o internecie w szkołach}
  \large
  \centering
  Platformy social-learningowe \\
  $\downarrow$ \\
  Wykorzystywanie ich w celu przepisania odpowiedzi do zadań, bez uprzedniego wykonania ich \\
  $\downarrow$ \\
  Nieprzyswojenie i nieprzetworzenie zadanego materiału \\
  $\downarrow$ \\
  Zaburzenie systemu nauczania \\
  $\downarrow$ \\
  Problem \\
  $\downarrow$ \\
  \textbf{Negatywna opinia nauczycieli o internecie jako ogóle}

\end{frame}

\begin{frame}
  \frametitle{Konsekwencje popularności serwisów social-learningowych}
  \begin{enumerate}
    \item Utrudnienie pracy nauczycielom
    \item Zmniejszenie wydajności systemu nauczania
    \item Niska jakość przyswajanego materiału
    \item Wypieranie rzetelnych źródeł wiedzy
  \end{enumerate}
  \begin{block}{Materiał}
    Tu: określone zagadnienie z podstawy programowej, które nauczyciel musi przerobić w ciągu roku szkolnego.
  \end{block}
\end{frame}

\section{Aplikacja}
\subsection{Przedstawienie aplikacji}

\begin{frame}
  \frametitle{ultraCALC: Aplikacja projektowa}
  Osiągnięte cele programistyczne:
  \begin{enumerate}
    \item Dynamiczne przeliczanie zmiennych \footnote[frame]{Algorytm, którego wynikiem jest wskazanie brakującej zmiennej oraz obliczenie jej.}
    \item Przetwarzanie definicji w czasie rzeczywistym \footnote[frame]{Przetworzenie surowych informacji, pochodzących z bazy danych, do interfejsu graficznego.}
    \item Dynamiczny i modularny interfejs
    \item Publikacja w sklepie Google Play \footnote[frame]{Domyślna platforma z aplikacjami na system Android -  \emph{play.google.com}.}
  \end{enumerate}
\end{frame}

\begin{frame}
  \frametitle{ultraCALC: Aplikacja projektowa}
  Aplikacja została stworzona, aby:
  \begin{enumerate}
    \item Móc zebrać rzetelne statystyki ze środowiska szkolnego dla poparcia tezy projektu
    \item Znaleźć alternatywę dla platform social-learningowych
    \item Mieć satysfakcję z napisania działającej aplikacji...
  \end{enumerate}
  \begin{block}{Teza}
    Współczesne rozwiązania technologiczne pomagają w rozwoju i edukacji
  \end{block}
\end{frame}

\begin{frame}
  \frametitle{Interfejs}
  \begin{columns}
  \begin{column}{0.4\textwidth}
     \begin{enumerate}
       \item Generator wzorów
       \item Jednostki
       \item Przelicznik wzorów
       \item Definicja
     \end{enumerate}
  \end{column}
  \begin{column}{0.6\textwidth}  %%<--- here
    \begin{figure}[p]
      \includegraphics[width=6.5cm]{screenshot}
    \end{figure}
  \end{column}
  \end{columns}
\end{frame}

\begin{frame}
  \frametitle{System nauki}
  \small
  \begin{table}
    \centering
    \begin{tabular}{ p{4cm} p{5cm} }
      \toprule
      Element & Rola w systemie \\
      \midrule
      Generator wzoru & Wzrokowe utrwalanie budowy wzoru \\
      Przelicznik wzorów & Utrwalanie definicji poprzez weryfikowanie otrzymanego wyniku  \\
      Definicja & Utrwalanie definicji poprzez \textit{czytanie}\\
      \bottomrule
    \end{tabular}
  \end{table}

  Skuteczność wzrasta wraz z częstotliwością wykorzystywania aplikacji przy robieniu zadań.

\end{frame}

\section{Doświadczenia}
\subsection{Przeprowadzanie testów w środowisku szkolnym}

\begin{frame}
\frametitle{Cel doświadczeń}
\begin{enumerate}
  \item Potwierdzić tezę
\end{enumerate}
\end{frame}

\section{Wyniki}

\begin{frame}
b
\end{frame}

\section{Podsumowanie}

\begin{frame}
  \frametitle{Możliwe rozwiązanie problemu}
\end{frame}

\end{document}
